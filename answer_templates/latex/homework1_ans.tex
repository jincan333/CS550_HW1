\documentclass[11pt]{article}
\usepackage{fullpage}
\usepackage{url}
\begin{document}
\thispagestyle{empty}
\parindent 0pt
\vfill
\large

\begin{center}
\LARGE{\bf \textsf{CS550: Massive Data Mining}}\\ {\bf \textsf{Homework 1}} 
\\*[4ex]
Due 11:59pm Monday, February 23, 2026\\
Please see the homework file for late policy
\end{center}

\pagebreak[4]
\begin{center}
\LARGE{\bf \textsf{Submission Instructions}} \\*[4ex]
\end{center}

\textbf{Honor Code } Students may have discussions about the homework with peers. However, each student must write down their solutions independently to show they understand the solution well enough in order to reconstruct it by themselves.  Students should clearly mention the names of all the other students with whom they have discussions about the homework. Directly using the code or solutions obtained from the web or from others is considered an honor code violation. We check all the submissions for plagiarism and take the honor code seriously, and we hope students to do the same. 

\vfill
\vfill

Discussions (People with whom you discussed ideas used in your answers): \\\\\\
On-line or hardcopy documents used as part of your answers: \\\\\\
\vfill

\vfill

I acknowledge and accept the Honor Code.\\*[3ex]
\bigskip
\textit{(Signed)}\hrulefill

If you are not printing this document out, please type your initials above.

\vfill
\vfill

\pagebreak[4]
\section*{Answer to Question 1}

\subsection*{Algorithm Description}

The problem is solved with a single MapReduce job.

\textbf{Map phase:} For each user $U$ with friend list $F = \{f_1, f_2, \ldots, f_n\}$, the mapper performs two kinds of emissions:
\begin{enumerate}
    \item For every pair $(f_i, f_j)$ in $F$ (i.e., two friends of $U$), emit $(f_i, (f_j, 1))$ and $(f_j, (f_i, 1))$. This records that $f_i$ and $f_j$ share $U$ as a mutual friend, so they are potential recommendations for each other.
    \item For every friend $f_i \in F$, emit $(U, (f_i, -1))$ to mark that $U$ and $f_i$ are already friends.
\end{enumerate}

\textbf{Reduce phase:} For each user $U$, the reducer aggregates all emitted values. It sums up the mutual-friend counts for each candidate and filters out candidates that are already direct friends (marked with $-1$). The remaining candidates are sorted in decreasing order of mutual-friend count, with ties broken by ascending user ID. The top 10 are output as recommendations.

\subsection*{Recommendations for Specified Users}

\begin{center}
\begin{tabular}{|c|l|}
\hline
\textbf{User} & \textbf{Recommendations} \\
\hline
924 & 439, 2409, 6995, 11860, 15416, 43748, 45881 \\
8941 & 8943, 8944, 8940 \\
8942 & 8939, 8940, 8943, 8944 \\
9019 & 9022, 317, 9023 \\
9020 & 9021, 9016, 9017, 9022, 317, 9023 \\
9021 & 9020, 9016, 9017, 9022, 317, 9023 \\
9022 & 9019, 9020, 9021, 317, 9016, 9017, 9023 \\
9990 & 13134, 13478, 13877, 34299, 34485, 34642, 37941 \\
9992 & 9987, 9989, 35667, 9991 \\
9993 & 9991, 13134, 13478, 13877, 34299, 34485, 34642, 37941 \\
\hline
\end{tabular}
\end{center}

\pagebreak[4]
\section*{Answer to Question 2(a)}

\pagebreak[4]
\section*{Answer to Question 2(b)}

\pagebreak[4]
\section*{Answer to Question 2(c)}

\pagebreak[4]
\section*{Answer to Question 2(d)}

\pagebreak[4]
\section*{Answer to Question 2(e)}

\pagebreak[4]
\section*{Answer to Question 3(a)}

\pagebreak[4]
\section*{Answer to Question 3(b)}

\pagebreak[4]
\section*{Answer to Question 3(c)}



\end{document}

